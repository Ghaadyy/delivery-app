\documentclass{article}
\usepackage{graphicx}
\usepackage{hyperref}

\usepackage[
    top=2.5cm,
    bottom=2.5cm,
    left=2.5cm,
    right=2.5cm
]{geometry}

\begin{document}

\input{title.tex}

\section*{Repository Link}

\href{https://github.com/Ghaadyy/delivery-app}{https://github.com/Ghaadyy/delivery-app}

\section*{Application Overview}
The application is designed to streamline food ordering and delivery, offering users a seamless experience to discover restaurants, order and track deliveries. Its primary features include:
\begin{itemize}
    \item Secure user authentication using JWT.
    \item Browse restaurants, add food items to the cart, and place orders.
    \item Mark your favorite restaurants for quick access in the future.
    \item Submit and view restaurant reviews.
    \item Track order history, including past and pending orders.
    \item Access detailed order information, such as the current status of an order.
    \item Order tracking based on your location and the restaurant's location.
    \item Rate delivery drivers to provide feedback.
    \item View and update personal details through the user profile.
\end{itemize}

\section*{Technical Details}
The application leverages the following tools, libraries, and design principles:
\begin{itemize}
    \item \textbf{Pattern:} MVVM (Model-View-ViewModel).
    \item Separation of concerns for scalability and testability.
\end{itemize}
\textbf{Technologies:}
\begin{itemize}
    \item \textbf{Programming Language:} Kotlin.
    \item \textbf{Libraries:}
          \begin{itemize}
              \item Retrofit for API calls.
              \item Room for local database.
              \item Jetpack Compose for a declarative UI.
              \item Google Play Services (GM) to access location data.
              \item LiveData and ViewModel for UI-related data.
              \item Osmdroid for the map rendering.
              \item JWTDecode for token authentication.
          \end{itemize}
    \item \textbf{Design Patterns:}
          \begin{itemize}
              \item Singleton for shared resources.
              \item Repository for data management.
          \end{itemize}
\end{itemize}


\section*{Challenges and Solutions}
During development, several challenges were encountered:
\begin{itemize}
    \item \textbf{Token persistence across app sessions}\\
          \textbf{Solution:} The JWT token was stored securely in SharedPreferences and validated using the JWTDecode library.
    \item \textbf{Efficient state management across activities}\\
          \textbf{Solution:} Shared ViewModel instances were used to maintain consistency between activities.
    \item \textbf{Integrating Compose with View-based UIs}\\
          \textbf{Solution:} A clear project structure allowed us to easily collaborate and integrate both UI technologies into the project.
\end{itemize}

\section*{Setup Instructions}
To set up and run the application, follow these steps:
\begin{enumerate}
    \item Clone the repository: \texttt{git clone \href{https://github.com/Ghaadyy/delivery-app}{https://github.com/Ghaadyy/delivery-app}}.
    \item Open the API in Visual Studio Code (Make sure you have .NET 8 installed).
    \item Install PostgreSQL.
    \item Resolve all dependencies and set up the database.
    \item Run the server.
    \item Open the project in Android Studio.
    \item Ensure that you have the required dependencies installed: \begin{itemize}
              \item Android SDK 34 is a working verision.
              \item Gradle 8.7 is a working version.
          \end{itemize}
    \item Build the project and resolve dependencies.
    \item Run the app on an emulator or physical device.
\end{enumerate}

\end{document}
